 \chapter{Conclusions And Future works} \label{ch:Discussion And Resuls}
 
\section{Conclusions}
In this thesis, we proposed a patient-adaptable ECG classification framework. The system has a two-staged hierarchical structure including a global classifier and personalized classifier. Global classifier is designed to filter the samples with severe distortion and abnormal waveforms by triggering red alarms. The samples classified as ``normal'' by global classifier are delivered to the deviation detection stage. In this stage, the personalized dynamic normal cluster is constructed and used to specify the normal range for each patient. By comparing the a sample with the personalized normal range, we use two joint conditions to decide if the sample is in a fuzzy state between normality and abnormality. If the sample fails to meet the joint conditions, a yellow alarm will be triggered to provide predictive information about upcoming abnormalities and the sample will be passed to the personalized classifier, which will label it as one of the three abnormal types. Whereas, the samples without detected deviation are further utilized to update personalized dynamic normal cluster.

In Chapter 3, a kernel-based nonlinear transformation is proposed to address the problem of cluster topology in original feature space. More specifically, the weighted combination of kernel functions are deployed in this method as a spatial transformation function. The ideal topology is formulated as two objective functions, so that the system is able to find the optimal coefficients of kernels by jointly optimizing these two functions. This non-convex multi-objective optimization is solved with MOPSO. In order to validate the improvement on spatial topology by introducing nonlinearities with kernels, we compared the Pareto front generated with linear combination of original features and Pareto front produced in mapped feature space with polynomial kernels. The result verifies that the kernel-based transformation allows more degree of freedom so that the topology can be further optimized according to the objective functions. Moreover, we applied this method on MITDB test data and obtained similar sensitivity and specificity as proposed in the literature. More importantly, the predictive capability of yellow alarms is analyzed. The performance is quantified by comparing prior probabilities and posterior probabilities for each type of yellow alarms. The comparison result shows that a promising improvement has been made by applying the nonlinear transformation. 

While the method in Chapter 3 demonstrated capacity of predicting upcoming abnormalities of ECG signal, it remains challenging to interpret the mechanisms of the systems and thus hindering the generalization of predictive warning in biomedical signal applications. Therefore, the main objective of Chapter 4 is to develop a deterministic spatial transformation function, which is able to achieve the ideal spatial topology at the same time. Thus, we proposed a novel spatial transformation specifically designed to reshape the feature space according to angles between cluster centroids. In this method, between-cluster cosine distances are optimized through orthogonalization of cluster centroids using spherical coordinate. Meanwhile, within-cluster variance is reduced by a piecewise mapping function composed with designed basis functions. The basis function proposed in Chapter 4 has the property of saturating at the boundaries, which is similar to sigmoid function but yet more flexible. An advantage of deploying such basis function is that the cluster geometry is preserved after spatial transformation. 
We implement this novel transformation in the patient-adaptable classification framework, the performance of this system is evaluated with classification and prediction results on the test dataset. The classification results show that by triggering yellow alarms through this method, specificity of abnormal types is improved. Especially for S (supraventricular) type, the proposed system performs better on identifying S class than all 5 methods in the literature.
Moreover, compared to the method proposed in Chapter 3, this method improves predictive capability for all abnormal classes but the improvement for type S is most significant. 

We also studied the time lag of between a yellow alarm and the subsequent real abnormality in Chapter 4. The result shows that most of the real abnormalities occurs within 10 beat after a yellow alarm. Generally speaking, the system has been proven to be efficient both in classification and prediction in this work.


\section{Future works}
In this research, we focused on improving two main drawbacks of automated ECG analysis in literature, namely, failure to adapt to the inter-patient variability and incapability of early detection and prediction. We proposed two methods for improving predictive capability. While the result shows the promising performance of designed system, further investigations can help on generalizing and improving the proposed system. The following tasks can be resolved as a continuation of this research:
 \begin{itemize}
\item Research on other kernel functions to improve the transformation for spatial topology optimization.
\item Investigate on the deterministic solution for the objective functions proposed in Chapter 3.
\item Assess the performance of proposed spatial transformation on other biomedical signals with similar properties as ECG signal.
\item  Improve the deterministic mapping function in Chapter 4 by including clusters size in mapping function.
\end{itemize}

