 \chapter{Conclusions And Future Works} \label{ch:Discussion And Resuls}
 
\section{Conclusions}
In this thesis, we consider an overlooked problem in the biomedical signal processing community. Several automated algorithms are designed by the biomedical engineering community to process physiological signals to assist health providers in diagnosing different disorders and making better therapy plans. However, an important concept of predicting severe heart conditions before their occurrence by processing physiological signals is not well studied. In this work, we developed a novel methodology as a primary step towards developing predictive diagnosis tools. We applied the proposed methodology into ECG signals to assess the power of the system in predicting upcoming heart problems.  
In this regard, we proposed a patient-adaptive ECG classification framework. The system has a two-staged hierarchical structure including a global classifier and a personalized classifier. Global classifier is designed to filter the samples with severe distortions represented by their abnormal waveforms by triggering red alarms. The samples classified as ``normal'' the by global classifier are delivered to the subsequent deviation detection module. In this stage, the personalized dynamic normal cluster is constructed and used to specify the normal range for each patient's ECG signal. By comparing a sample with its personalized normal range, we use two joint conditions to decide if the sample is in a fuzzy state between the normality and abnormality conditions. If the sample fails to meet both of these conditions, a yellow alarm is triggered to provide predictive information about upcoming abnormalities. The samples that show considerable deviations from their ground normal are passed to the personalized classifier, which will label it as one of the three abnormal types. On the other hand, the samples without considerable deviations are further utilized to update the personalized dynamic normal cluster.

In Chapter 3, a kernel-based nonlinear transformation is proposed to address the problem of cluster topology in the original feature space. More specifically, the weighted combination of kernel functions are deployed in this method as a spatial transformation function. The desired topology is formulated in terms of two objective functions, so that the system is able to find the optimal coefficients of kernels by jointly optimizing these two functions. This non-convex multi-objective optimization is solved with MOPSO. In order to validate the improvement on spatial topology by introducing nonlinearities with kernels, we compared the Pareto front generated with linear combination of the original features and Pareto front produced in the transformed feature space with polynomial kernels. The results confirm that the kernel-based transformation allows more degree of freedom in optimizing the topology in optimizing according to the objective functions. Moreover, we applied this method to MITDB test dataset and obtained similar sensitivity and specificity results as proposed in the literature. More importantly, the predictive capability of yellow alarms is analyzed. The performance is quantified by comparing the prior and posterior probabilities for each type of yellow alarms. The comparison results show that a promising improvement has been achieved by applying the nonlinear transformation. 

While the method in Chapter 3 demonstrates a capacity of predicting upcoming abnormalities of ECG signals, it remains challenging to interpret the mechanisms of the proposed system and thus hindering the generalization of the proposed predictive warning methodology to similar  biomedical signal processing applications. Therefore, the main objective of Chapter 4 is to develop a deterministic spatial transformation function, which is able to achieve the desired spatial topology with a more tractable analytical approach. Thus, we proposed a novel spatial transformation specifically designed to reshape the feature space according to angles between cluster centroids. In this method, between-cluster cosine distances are optimized through orthogonalization of cluster centroids using spherical coordinate. Meanwhile, within-cluster variance is reduced by a piecewise mapping function composed with designed basis functions. The basic function proposed in Chapter 4 has the property of saturating at the boundaries, which is similar to the inverse of logit function but yet more flexible. An advantage of deploying such basics function is that the cluster geometry is preserved after spatial transformation. 
We implement this novel transformation in the patient-adaptive classification framework, the performance of this system is evaluated with classification and prediction results on the test dataset. The classification results show that by triggering yellow alarms through this method, the \textit{specificity} of abnormal types is improved. Especially for the type S (supraventricular), the proposed system performs better in identifying class S than all 5 similar benchmark methods reported in the literature.
Moreover, compared to the method proposed in Chapter 3, this method improves the predictive capability of the developed system for all abnormal classes but the improvement for type S is most significant. 

We also studied the time lag of between a yellow alarm and the subsequent real abnormality in Chapter 4. The result shows that most of the real abnormalities occur within 10 beat after a yellow alarm. Generally speaking, the system has been proven to be efficient both in classification and prediction in this work. In short, this study suggest that predictive modeling of physiological signals can be used as alarming hints for upcoming health conditions, which can have a wide range of applications in developing wearable biosensors,  automated diagnosis tools to assist patents and physicians in predicting health problems, This methodology have a great potential to impact the emerging research fields of smart health and smart cities.  


\section{Future Works}
In this research, we focused on improving two main drawbacks of automated ECG analysis in the literature, namely, the failure in capturing the inter-patient variability and the incapability of early detection and prediction. We proposed two methods for improving predictive capability. While the results show the promising performance of the designed system, further investigations can help to improve and generalize the proposed system to other types of biomedical signals. The following tasks can be considered as some future direction enabled by this research:
 \begin{itemize}
\item Investigate other kernel functions to improve the transformation for spatial topology optimization.
\item Seek deterministic solutions and develop more systematic performance analysis for the objective functions proposed in Chapter 3.
\item Assess the performance of the proposed spatial transformation on other biomedical signals with similar properties, such as EEG, EMG, and EOG.
\item  Improve the deterministic mapping function in Chapter 4 by including clusters size in the mapping function.
\end{itemize}

