 \chapter{Conclusions And Future works} \label{ch:Discussion And Resuls}
 
\section{Conclusions}
In this thesis, we a patient-adaptable ECG classification framework. The system has a two-staged hierarchical classifier structure including Global Classifier and Personal Classifier. While Global Classifier is designed to filter the signal with severe distortion and abnormal waveforms by triggering red alarms and pass other samples to the deviation detection stage. In this stage, the personal dynamic normal cluster is constructed and used to specify the normal range for each patient. By comparing the current sample and personalized normal range, this module decides if a yellow alarm will be triggered to provide predictive information about upcoming abnormalities. If a sample is detected with deviation towards abnormal clusters, it will be passed to the Personal Classifier and labeled as one of the three abnormal types. Whereas samples without deviation are further feed back to personal dynamic normal cluster to update the classification system about the newest personal normal range.

In Chapter 3, a kernel based nonlinear transformation is proposed to address the problem of cluster topology in original feature space. Inspired by Support Vector Machine, kernel functions are deployed in this method as a spatial transformation function. The target topology is formulated as two objective functions so that by tuning the parameters in kernel function, the system is able to select the best transformation for the following predicting stage. This non-convex multi-objective optimization is solved with Multi-Objective Particle Swarm Optimization. In order to validate improvement by using high order kernel function, we compared the Pareto front generated with linear combination of original features and mapped high order features with polynomial kernel. The result verifies that applying high order kernel function allows more degree of freedom so that the topology can be further optimized according to objective functions. Having this concept proved, we applied this method on MITDB test data and obtained similar sensitivity and specificity as proposed in the literatures. More importantly, the predicting capacity of yellow alarms are analyzed. The performance is quantified by comparing prior probability and posterior probability giving the types of yellows alarm. The comparison result shows that a promising improvement has been made by applying the nonlinear transformation. 

While the method in Chapter 3 demonstrated capacity of predicting upcoming abnormalities, it's challenging to interpret the mechanisms of the systems and thus hindering the generalization of predictive warning to other applications of biomedical signals. Therefore, the main object of Chapter 4 is developing a classification system with abnormality predicting capacity based on spatial topology studied in Chapter 3. In Chapter 4, we proposed a novel spatial transformation specifically designed to reshape the feature space according to angles between cluster center. In this method, between cluster cosine distance are optimized through orthogonalization in spherical coordinate space and within cluster variance is reduced by a mapping function which is fitted piecewise with a basis function. The basis function proposed in the chapter has the feature of saturating at the boundaries, similar to sigmoid function but more flexible. An advantage of deploying such basis function is that the cluster geometry may be preserved after spatial transformation. With this novel module integrated in the patient-adaptable classification framework, the performance of this system is evaluated through classification and prediction results on the test set data. The classification results show that by triggering yellow alarm through this method, specificity of abnormal types is improved. Especially for Supraventricular types, the proposed system performs better than all 5 methods in the literature. The same conclusion holds for prediction performance. Compared to the method proposed in Chapter 3, this method improves predicting capacity for all abnormal classes and the most significant improvement is for type S. Moreover, we also studied the time delay of real abnormalities following a yellow alarm. It's proved that most of the real abnormalities occurs within 10 beat after a yellow alarm. Generally speaking, the system is proved to be efficient both in classification and prediction.


\section{Future works}
In this research, we focused on two challenges of ECG classification, namely, inter-patient variation and anomaly prediction. The framework of patient-adaptable classier includes both features. The methods of improving prediction accuracy are proposed and studied. While the result shows the efficiency of designed system, further improvements can be made through research. The following tasks can be resolved as a continuation of this research:
 \begin{itemize}
\item Research on other kernel functions which is potentially a better transformation for spatial topology optimization.
\item Investigate on the deterministic solution for the objective functions proposed in Chapter 3.
\item Assess the performance of proposed spatial transformation on other biomedical signals with similar characters as ECG signal
\item  Improve the deterministic mapping function in Chapter 4 by including the variance of individual clusters into function parameters
\end{itemize}

