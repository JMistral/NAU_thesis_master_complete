 \chapter{Conclusions And Future Works} \label{ch:Discussion And Resuls}
 
\section{Conclusions}
In this thesis, we considered an overlooked problem in the biomedical signal processing community, which is the lack of prediction power of automated biomedical signal processing methods in predicting severe health abnormalities ahead of time through processing mild abnormalities in the test signal. Several automated algorithms have been designed by the biomedical engineering community to process physiological signals to assist health providers in diagnosing different disorders and making better therapy plans. However, an important concept of predicting severe heart conditions before their occurrence by processing physiological signals is not well studied. In this work, we developed a novel methodology as a primary step towards developing predictive diagnosis tools. We applied the proposed methodology into ECG signals to assess the power of the system in predicting upcoming heart problems.  
In this regard, we proposed a patient-adaptive ECG classification framework. The system has a two-staged hierarchical structure including a \textit{global classifier} and a \textit{personalized classifier}. The \textit{global classifier} is designed to filter out the samples with severe distortions represented by their abnormal waveforms by triggering red alarms. The samples classified as ``normal'' the by \textit{global classifier} are delivered to the subsequent deviation detection module. In this stage, the \textit{personalized dynamic normal cluster} is constructed and used to specify the normal range for each patient's ECG signal. By comparing a sample with its personalized normal range, we use two joint conditions to decide if the sample is in a fuzzy state between the normality and abnormality conditions. If the sample fails to meet both of these conditions, a \textit{yellow alarm} is triggered to provide predictive information about upcoming abnormalities. The samples that show considerable deviations from their ground normal are passed to the \textit{personalized classifier} to label it as one of the three abnormal types, whichever is more likely. On the other hand, the samples without considerable deviations are confirmed to be normal and are further utilized to update the \textit{personalized dynamic normal cluster}.

In chapter 3, a \textit{kernel}-based nonlinear transformation is proposed to manipulate the clustering topology in the original feature space. More specifically, a weighted combination of kernel functions are deployed in this method to implement a spatial transformation function. The desired topology is formulated in terms of two objective functions, so that the system is able to find the optimal coefficients of kernels by jointly optimizing these two functions. This non-convex multi-objective optimization is solved with a method based on MOPSO. In order to validate the improvement on spatial topology by introducing nonlinearities with kernels, we compared the Pareto front generated with the linear combination of the original features and the Pareto front produced in the transformed feature space with the utilized polynomial kernels. The results confirm that the kernel-based transformation allows more degree of freedom in optimizing the clustering topology according to the representative objective functions. Moreover, we applied this method to MITDB test dataset and obtained similar sensitivity and specificity results as proposed in the literature. More importantly, the predictive capability of \textit{yellow alarms}, as a unique feature of the proposed method, is analyzed. The performance is quantified by comparing the prior and posterior probabilities of each abnormality class after observing a \textit{yellow alarm} of the same type. The comparison results show that a promising improvement has been achieved by applying the nonlinear transformation. 

While the method in chapter 3 demonstrates a capacity of predicting upcoming abnormalities of ECG signals, it remains challenging to interpret the mechanisms of the proposed system and thus hindering the generalization of the proposed predictive warning methodology to similar  biomedical signal processing applications. Therefore, the main objective of chapter 4 is to develop a deterministic spatial transformation function, which is able to achieve the desired spatial topology with a more tractable analytical approach. Thus, we proposed a novel spatial transformation specifically designed to reshape the feature space according to angles between cluster centroids using the customized inverse of \textit{logit} functions. In this method, the between-cluster cosine distances are optimized through orthogonalization of cluster centroids using spherical coordinates. Meanwhile, the within-cluster variance is reduced by a piecewise mapping function consisting of the previously designed basis functions. The basic function proposed in chapter 4 has the property of saturating at the boundaries, which is similar to the inverse of logit function but is yet more flexible. An advantage of deploying such basic functions is that the clustering geometry is preserved after spatial transformation. 
We implement this novel transformation in the patient-adaptive classification framework. The performance of this system is evaluated with classification and prediction results on the test dataset. The classification results show that by triggering \textit{yellow alarms} through this method, the \textit{specificity} \textcolor{black}{as well as the performance consistency }of abnormal types is improved   \textcolor{black}{compared with the method proposed in chapter 3 as we can observe higher \textit{specificities} and lower IQRs. The improvement on type V (\textit{ventricular}) is most notable, as the median of accuracy is improved from 86.11\% to 96.17\% and the IQR is lowered 22.37\%.} 
\textcolor{black}{Comparing with the methods in the literature, the proposed method in chapter 4 is remarkable as well. Especially for the type S (\textit{supraventricular}), the proposed system performs better in identifying class S than all 5 similar benchmark methods reported in the literature.}

We also studied the impact of the time lag between a \textit{yellow alarm} and the subsequent real abnormality in chapter 4. The results show that most of the real abnormalities occur within 10 samples after a \textit{yellow alarm}. Overall, the system has been proven to be efficient both in classification and prediction aspects in this work. In short, this study suggests that predictive modeling of physiological signals can be used as alarming hints for upcoming health conditions, which can have a wide range of applications in developing wearable biosensors,  automated diagnosis tools to assist patents and physicians in predicting health problems, This methodology have a great potential to impact the emerging research fields of smart health and smart cities.  


\section{Future Works}
In this research, we focused on improving two main drawbacks of automated ECG analysis in the literature, namely, the failure in capturing the inter-patient variability and the incapability of early detection and prediction. We proposed two methods for improving predictive capability. While the results show the promising performance of the designed system, further investigations can help to improve and generalize the proposed system to other types of biomedical signals. The following tasks can be considered as some future directions enabled by this research:
 \begin{itemize}
\item investigate other kernel functions to improve the transformation for spatial topology optimization;
\item seek deterministic solutions and develop more systematic performance analysis for the objective functions used to assess the \textit{separability} and \textit{symmetry} of clustering geometry in transformed feature space;
\item assess the performance of the proposed spatial transformation on other biomedical signals with similar properties, such as EEG, EMG, and EOG;
\item  improve the deterministic mapping function by including the size of clusters in the mapping function;
\item \textcolor{black}{integrate the measurements from other wearable body sensors (such as accelerometer, temperature etc.) in the monitoring system to capture the impact of environmental condition on temporal variation of signals;
}
\item \textcolor{black}{design a mechanism to correct the false \textit{red alarms} triggered by \textit{global classifier} and further improve the classification performance.
}
\end{itemize}

