
\begin{center}

\vfill

%{\large Predictive Modeling of Biomedical Signal based on Spatial Topology Based Feature Space Transformation }
{\large Remote Heart Monitoring: A Predictive Modeling Approach for Biomedical Signal Processing }


Jiaming Chen

{(ABSTRACT)}

%\vfill



\vfill

\end{center}

\textcolor{blue}{Smart healthcare is an emerging field with numerous research projects devoted to design electronic devices, computer technologies and platforms aiming at facilitate technology-based health service at lower costs. Biomedical signal can directly reflect the information regarding patient health and has therefore been frequently investigated. The essence of biomedical signal analysis system is to process signals and build statistical or machine learning model to provide informative result about the health status of patients. While the majority of methods in literature focus on improving classification performance on pooled dataset, the predictive modeling of biomedical signal is rarely emphasized.} %Biomedical signal classification has been frequently investigated by researchers in the past decades. Various classification systems have been proposed and evaluated by classification metrics. The essence of most methods, is to analyze test signals using reference models constructed based on a collection of healthy and abnormal signals. While system performances in terms of classifying signal samples increased significantly with modern machine learning algorithms such as recurrent neural network, there are one important factor in biomedical signal processing which is rarely studied in recent research. 
 In this work, we go one step beyond the conventional methods and intend to predict potential upcoming abnormalities before their occurrence. The objective is to build a patient-specific model and identify minor deviations from the normal signal, which can be indicative of potential upcoming significant deviations. 

\textcolor{blue}{To enable an accurate prediction on different data, two spatial transformation methods in which feature space are reshaped according to designed topology are proposed together with a patient-adaptable classification framework.} %To facilitate a sound deviation analysis, a controlled spatial transformation is proposed to reshape the signal geometry in the feature space, such that the abnormality classes symmetrically surround the normal class. 
We applied the developed algorithms on Electrocardiogram (ECG) signals and the results confirm the \textcolor{blue}{effectiveness} of the proposed method in predicting upcoming heart abnormalities before their occurrence. For instance, the probability of \textcolor{blue}{observing a} specific abnormality class increases by 10\% after \textcolor{blue}{triggering} a yellow alarm of the same type. This approach is general and has the potential to be applied to a wide range of physiological signals.


